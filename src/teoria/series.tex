\documentclass[../main.tex]{subfiles}

\begin{document}
En este capítulo presentamos un método para obtener soluciones a ecuaciones
diferenciales lineales en forma de series de potencias. Para ello, no queda otra
opción que restringirnos a soluciones \emph{analíticas}, es decir, funciones que
se puedan expresar de esta manera.

El método se entiende mejor con un ejemplo:

\begin{example}
  Expresar las soluciones a la ecuación
  \begin{equation} \label{eq:eqpot}
    x''-2tx'-2x=0
  \end{equation}
  como series de potencias.

  Buscamos soluciones analíticas con radio de convergencia \(\geq \rho\) y
  estamos interesados en el rango de tiempos \(t\) tal que \(|t-t_0| <
  \rho\). En este ejemplo tomamos \(t_0 = 0\), por lo que las soluciones serán
  de la forma
  \begin{equation}
    \label{eq:solpot}
    x(t) = \sum_{n=0}^\infty a_nt^n, \quad |t| < \rho
  \end{equation}

  Nuestro objetivo se reduce, entonces, a dar ``ecuaciones'' que caractericen la
  sucesión \((a_n)_n\), de forma que, al resolverlas, obtengamos la función \(x(t)\).

  Derivando en~\eqref{eq:solpot}, llegamos a las series de potencias
	\begin{align*}
		x'(t) &= \sum_{n=1}^\infty na_nt^{n-1} = \sum_{n=0}^\infty (n+1)a_{n+1}t^n, \\
		x''(t) &= \sum_{n=2}^\infty n(n-1)a_nt^{n-2} = \sum_{n=0}^\infty (n+2)(n+1)a_{n+2}t^n,
	\end{align*}
donde ambas series de potencias tienen radio de convergencia \(\geq \rho\).

Teniendo esto en cuenta y volviendo a~\eqref{eq:eqpot}, tenemos
\[x''-2tx'-2x=0 \leftrightsquigarrow \sum_{n=0}^\infty (n+2)(n+1)a_{n+2}t^n -
  2t \sum_{n=0}^\infty (n+1)a_{n+1}t^n - 2 \sum_{n=0}^\infty a_nt^n = 0\]
o, equivalentemente,
\[\sum_{n=0}^\infty (n+2)(n+1)a_{n+2}t^n - \sum_{n=0}^\infty
  2(n+1)a_{n+1}t^{n+1} - \sum_{n=0}^\infty 2a_nt^n = 0.\]
Escribimos esto como una única serie de potencias:
\[\underbrace{2a_2-2a_0}_{n=0} +
  \sum_{n=1}^\infty[(n+1)(n+2)a_{n+2}-2na_n-2a_n]t^n = 0\]
que, como es idénticamente nula, tiene todos sus coeficientes iguales a 0:
\begin{equation}
  \label{eq:seqpot}
  \begin{cases}
    a_2 = a_0 \\
    (n+1)(n+2)a_{n+2} = 2(n+1)a_n, & n \geq 1
  \end{cases} \iff
  a_{n+2} = \frac{2}{n+2} a_n, \qquad n \geq 0
\end{equation}

En resumen,~\eqref{eq:solpot} es solución de ~\eqref{eq:eqpot} si y sólo si
\((a_n)_n\) verifica estas condiciones. Una solución queda unívocamente
determinada al fijar \(x(0)\) y \(x'(0)\), que en este caso corresponden,
respectivamente, a \(a_0\) y \(a_1\). Como ya sabemos, una base del espacio de
soluciones de~\eqref{eq:eqpot} se obtiene al resolver~\eqref{eq:seqpot} para los
casos
\[
  \begin{cases}
    a_0 = 1 \\ a_1 = 0
  \end{cases} \quad \text{y} \quad
  \begin{cases}
    a_0 = 0 \\ a_1 = 1
  \end{cases}
\]
\begin{itemize}
\item Caso \(a_0 = 1,\ a_1 = 0\):
  \[a_2 = 1, \quad a_3 = 0, \quad a_4 = \frac{1}{2}, \quad a_5 = 0, \quad a_6 =
    \frac{1}{3}a_4 = \frac{1}{3!}, \quad a_7 = 0, \quad a_8 = \frac{1}{4}a_6 =
    \frac{1}{4!}, \quad \cdots\]
  Resulta evidente que
  \[a_{2k} = \frac{1}{k!} \quad \text{y} \quad a_{2k+1} = 0,\]
  y es inmediato probarlo por inducción.

  En este caso la solución es
  \[x(t) = \sum_{k=0}^\infty \frac{t^{2k}}{k!} = e^{t^2}, \qquad \rho =
    +\infty\]
\item Caso \(a_0 = 0,\ a_1 = 1\):
  \[a_2 = 0, \quad a_3 = \frac{2}{3}, \quad a_4 = 0, \quad a_5 = \frac{2}{5}a_3
    = \frac{2^2}{3 \cdot 5}, \quad a_6 = 0, \quad a_7 = \frac{2}{7}a_5 =
    \frac{2^3}{3 \cdot 5 \cdot 7}, \quad \cdots\]
  La fórmula general es ahora
  \[a_{2k} = 0 \quad \text{y} \quad a_{2k+1} = \frac{2^k}{\prod_{i=1}^k
      (2i+1)},\]
  e igual que antes es inmediato probarlo por inducción.

  En este caso, \(\sum_{n=0}^\infty a_nt^n\) no es una función elemental, aunque
  es analítica con radio de convergencia \(\rho = +\infty\).
\end{itemize}
\end{example}

Para terminar con el capítulo, recogemos las ideas generales en un teorema:
\begin{theorem}
  Supongamos que \(Q(t)\), \(R(t)\) y \(P(t)\) se pueden escribir como series
  de potencias alrededor de \(t_0\) con radio de convergencia \(\rho\):
  \[Q(t) = \sum_{n=0}^\infty q_n(t-t_0)^n, \quad R(t) = \sum_{n=0}^\infty
    r_n(t-t_0)^n, \quad P(t) = \sum_{n=0}^\infty p_n(t-t_0)^n, \qquad |t-t_0| <
    \rho\]
  Entonces, \emph{todas} las soluciones de la ecuación
  \begin{equation}
    \label{eq:thmpot}
    x'' + Q(t)x' + R(t)x = P(t)
  \end{equation}
  se pueden escribir también como series de potencias alrededor de
  \(t_0\) con radio de convergencia \(\rho\), es decir,
  \[x(t) = \sum_{n=0}^\infty a_n(t-t_0)^n, \qquad |t-t_0| < \rho.\]
  Además, en este caso, para obtener los coeficientes \(a_n\) se puede razonar
  como en el ejemplo: se asume \(x(t) = \sum_{n=0}^\infty a_n(t-t_0)^n\), se
  escribe~\eqref{eq:thmpot} como serie de potencias y se igualan sus
  coeficientes a 0. Esto da unas ecuaciones que caracterizan la sucesión
  \((a_n)_n\).
\end{theorem}

\begin{remark}
  Para expresar el producto de dos series de potencias como serie de potencias
  se usa el \emph{producto de Cauchy} o \emph{convolución}:
  \[\left( \sum_{n=0}^\infty A_n (t-t_0)^n \right) \cdot
    \left( \sum_{n=0}^\infty B_n (t-t_0)^n \right) =
    \sum_{n=0}^\infty C_n (t-t_0)^n,\]
  siendo
  \[C_n = \sum_{k=0}^n A_kB_{n-k}.\]
\end{remark}
\end{document}
