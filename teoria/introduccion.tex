\documentclass[../main.tex]{subfiles}

\begin{document}

\section{Primeras definiciones}

\begin{definition}
	Una ecuación diferencial es una ecuación de la forma 
	\[F(t, x, x', x'', \dots, x^{n)}) = 0\]
	donde \(F : \Omega \to \R\) con \(\Omega \subset \R^{n + 2}\) abierto.
\end{definition}

Se dice que \(x(t)\) es solución de una ecuación diferencial en el intervalo 
\(I \subset \R\) si \(x : I \to \R\) y
\[F(t, x(t), x'(t), x''(t), \dots, x^{n)}(t)) = 0, \quad \forall t \in I\]
y se cumple además que
\begin{enumerate}[i)]
	\item \(x\) tiene todas las derivadas hasta orden \(n\) y son continuas.
	\item \((t, x(t), x'(t), \dots, x^{n)}(t)) \in \Omega, \quad \forall t \in I\).
\end{enumerate}

Supondremos que la ecuación diferencial está dada en forma normal,
\[x^{n)} = f(t, x', x'', \dots, x^{n - 1)})\]
llamamos \(\Omega \subset \R^{n + 1}\) al dominio de \(f\), esto se puede hacer
localmente sin pérdida de generalidad gracias al teorema de la función
implícita.

\begin{definition}
	Llamaremos a \(n\) el orden de la ecuación diferencial.
\end{definition}

\begin{definition}
	La ecuación es lineal si \(f\) es de la forma
	\[f(t) = a_{n - 1}(t)x^{n - 1)} + a_{n - 2}(t)x^{n - 2)} + 
		\dots + a_1(t)x' + a_0(t)x + b(t)\]
\end{definition}

\section{Ejemplos de modelado}

\subsection{Desintegración de sustancias radiactivas}

En este caso tenemos que \(x(t)\) es
la masa en un tiempo \(t\) de un elemento radiactivo. La clave que nos permite
modelar este fenómeno es que la probabilidad de desintegración de un átomo es la
misma para todos los átomos de la muestra, independientemente de la masa del
material.

Nos interesa la masa en función del tiempo relativa, esto es
\[\frac{\Delta x}{x \Delta t} = \frac{x(t + \Delta t) - x(t)}{x(t) \Delta t}
	= -\lambda, \quad \lambda > 0\]
la última igualdad se deduce de lo dicho anteriormente. Llamaremos a 
\(\Delta t\) la tasa de crecimiento media. Tomando el límite \(\Delta t \to 0\)
nos queda \(x'(t)/x(t) = -\lambda\) para todo \(t\). Esto nos da una ecuación
diferencial lineal de primer orden (de coeficientes constantes).

En este caso, \(x\) es una masa, por lo que no puede ser negativa, por tanto
\[f(t, x) = -\lambda x\] con \(f : \Omega \to \R\) y \(\Omega = \R \times
(0, +\infty)\).

Esta ecuación diferencial es resoluble, para ello tomamos primitivas en
\[\frac{x'(t)}{x(t)} = -\lambda \implies \log(x) = -\lambda t + c
	\implies x = e^c e^{-\lambda t} = k e^{-\lambda t}\]
con lo que nos queda \(x = k e^{-\lambda t}\), llamamos a esta fórmula solución
general de la ecuación, ya que nos da todo el conjunto de soluciones variando
\(k\).

Observamos que el conjunto de soluciones en un espacio vectorial de dimensión 1,
esto se debe a que la ecuación es lineal y de dimensión 1. Esto intuitivamente
se debe a que hemos acotado el espacio vectorial de dimensión infinita de todas
las derivadas de la función a uno de dimensión finita.

\subsection{Movimiento armónico simple}

Vemos ahora otro ejemplo de modelado más complicado, el movimiento del péndulo.
El punto de masa \(m\) es el peso del péndulo y llamaremos \(\theta\) al ángulo
formado entre la recta perpendicular al plano sobre el que esta apoyado el
péndulo y que parte de su punto de apoyo y la cuerda. Denotaremos la longitud
de la cuerda por \(l\) Recordamos que \(F = ma\) donde \(a\) es la aceleración. 
Tenemos entonces que
\[-m g \sin\theta = m l \theta''\]
por tanto
\[\theta'' = -\frac{g}{l} \sin\theta\]
puesto que \(\theta\) aparece dentro del seno, esta ecuación es no lineal y de
orden 2, puesto que aparece la derivada segunda de \(\theta\). En este caso 
\(f(t, \theta, \theta') = -\frac{g}{l} \sin\theta\) y \(\Omega = \R \times
(-\pi, \pi) \times \R\). Si \(\theta\) es suficientemente próximo a \(0\)
podemos aproximar \(\sin\theta\) a \(\theta\) con lo que tenemos la ecuación
lineal de segundo orden
\[\theta'' = -\frac{g}{l}\theta.\]

Esta es la ecuación del movimiento armónico simple. Su solución general, que, 
recordamos, es el conjunto de todas las soluciones es
\[\Theta(t) = C_1 \sin \omega t + C_2 \cos \omega t\]
donde \(\omega^2 = g/l\), demostraremos esto más adelante. Este conjunto de
soluciones forma un espacio vectorial de dimensión 2. Demostraremos también que
el conjunto de soluciones de una ecuación lineal de orden \(n\) es un espacio
vectorial, o afín si existe un término independiente, de orden \(n\).

\subsection{Problema del valor inicial (PVI)}

Dada una ecuación diferencial en forma normal
\[x^{n)} = f(t, x, x', \dots, x^{n - 1)})\]
nos dan los valores de la función y sus derivadas en un tiempo determinado
\(t_0\), \(x(t_0) = x_0, x'(t_0) = x_1, \dots, x^{n - 1)}(t_0) = x_{n - 1}\).
Estas condiciones determinan una única solución para la ecuación diferencial
original.

\subsection{Datación por carbono 14 (Premio Nobel de Química 1960)}

Existen diversos isótopos del carbono, entre ellos el carbono 12 el cual es
estable y el carbono 14, que es el isótopo radiactivo más común. Podemos
entonces usar el primer ejemplo con \(\lambda \sim \log 2 / 5730\) medido en
años. La proporción de carbono 12 y carbono 14 es constante en la atmósfera. El
carbono 14 al ser radiactivo debería desintegrarse y disminuir su concentración
en la atmósfera pero este equilibrio se mantiene gracias a los rayos cósmicos
que alcanzan la tierra. Debido a la fotosíntesis y a la cadena alimenticia esta
proporción se mantiene constante en los seres vivos, énfasis en vivos. Al morir
un ser vivo el carbono 12 se mantiene mientras que el carbono 14 disminuye en
concentración, debido a su naturaleza radioactiva, aun ritmo que depende de
\(\lambda\). Debido a esto analizando la proporción entre carbono 12 y 14 en
material orgánico (madera, hueso, etc.) y comparándolo con uno vivo es posible
determinar la edad del resto orgánico analizado. Vemos algunos ejemplos
concretos.

\begin{example}
	Si la cantidad de carbono 14 en un microorganismo es de \(10^{-6}\) gramos 
	¿Qué cantidad habrá 3000 años después? 
\end{example}

Se entiende que el ciclo de vida del organismo
es despreciable respecto a la edad que debemos datar. Tenemos que 
\(x' = -\lambda x\), a tiempo \(0\), cuando el organismo muere, tenemos que
\(x(0) = 10^{-6}\). Se trata de un problema de valor inicial, ahora solo tenemos
que encontrar \(x(3000)\). Empezamos resolviendo la ecuación diferencial
\[x(t) = k e^{-\lambda t} = k e^{-\frac{\log 2}{5730} t}\]
por lo que hemos visto antes \(k = x(0) = 10^{-6}\) con lo que
\[x(t) = 10^{-6} e^{-\frac{\log 2}{5730} t}\]
podemos obtener ahora el resultado \(x(3000) = 6,95 \cdot 10^{-7}\) gramos.

Normalmente no sabemos la cantidad de carbono presente en un organismo antes de
su muerte, vemos ahora un ejemplo más cercano a la realidad. 

\begin{example}
	En una excavación en Nippur, Babilonia, en 1950 el carbón vegetal de la viga
	de un techo dio \(4,09\) desintegraciones por minuto y gramo, mientras que 
	la madera viva da \(6,68\) desintegraciones, también por minuto y gramo. 
	Nos preguntamos en qué año se construyó en edificio.
\end{example}

Llamaremos \(R(t)\) a la tasa de desintegración total
\(R(t) = \lambda x(t) = \lambda x(0) e^{-\lambda t}\), donde \(x(0)\) es la
cantidad de carbono 14 en el trozo de madera cuando fue cortada y \(t\) el 
tiempo transcurrido desde que se corta el árbol hasta 1950. Buscamos entonces
el año de construcción \(R(0) = \lambda x(0)\), tenemos que
\[\frac{R(t)}{R(0)} = 10^{-6} e^{-\frac{\log 2}{5730} t}\]
sustituimos \(R(t) = 4,09\) y \(R(0) = 6,68\) con lo que \(t = 4055\) años. Así
concluimos que el edificio se construyó en el \(2105\) a.C.

\end{document}
