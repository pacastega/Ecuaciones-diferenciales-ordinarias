\documentclass[../main.tex]{subfiles}

\begin{document}

\section{Primeras definiciones}

\begin{definition}
	Una ecuación diferencial es una ecuación de la forma 
	\[F(t, x, x', x'', \dots, x^{n)}) = 0\]
	donde \(F : \Omega \to \R\) con \(\Omega \subset \R^{n + 2}\) abierto.
\end{definition}

Se dice que \(x(t)\) es solución de una ecuación diferencial en el intervalo 
\(I \subset \R\) si \(x : I \to \R\) y
\[F(t, x(t), x'(t), x''(t), \dots, x^{n)}(t)) = 0, \quad \forall t \in I\]
y se cumple además que
\begin{enumerate}[i)]
	\item \(x\) tiene todas las derivadas hasta orden \(n\) y son continuas.
	\item \((t, x(t), x'(t), \dots, x^{n)}(t)) \in \Omega, \quad \forall t \in I\).
\end{enumerate}

Supondremos que la ecuación diferencial está dada en forma normal,
\[x^{n)} = f(t, x', x'', \dots, x^{n - 1)})\]
llamamos \(\Omega \subset \R^{n + 1}\) al dominio de \(f\), esto se puede hacer
localmente sin pérdida de generalidad gracias al teorema de la función
implícita.

\begin{definition}
	Llamaremos a \(n\) el orden de la ecuación diferencial.
\end{definition}

\begin{definition}
	La ecuación es lineal si \(f\) es de la forma
	\[f(t) = a_{n - 1}(t)x^{n - 1)} + a_{n - 2}(t)x^{n - 2)} + 
		\dots + a_1(t)x' + a_0(t)x + b(t)\]
\end{definition}

\section{Ejemplos de modelado}

Desintegración de sustancias radioactivas. En este caso tenemos que \(x(t)\) es
la masa en un tiempo \(t\) de un elemento radioactivo. La clave que nos permite
modelar este fenómeno es que la probabilidad de desintegración de un átomo es la
misma para todos los átomos de la muestra, independientemente de la masa del
material.

Nos interesa la masa en función del tiempo relativa, esto es
\[\frac{\Delta x}{x \Delta t} = \frac{x(t + \Delta t) - x(t)}{x(t) \Delta t}
	= -\lambda, \quad \lambda > 0\]
la última igualdad se deduce de lo dicho anteriormente. Llamaremos a 
\(\Delta t\) la tasa de crecimiento media. Tomando el límite \(\Delta t \to 0\)
nos queda \(x'(t)/x(t) = -\lambda\) para todo \(t\). Esto nos da una ecuación
diferencial lineal de primer orden (de coeficientes constantes).

En este caso, \(x\) es una masa, por lo que no puede ser negativa, por tanto
\[f(t, x) = -\lambda x\] con \(f : \Omega \to \R\) y \(\Omega = \R \times
(0, +\infty)\).

Esta ecuación diferencial es resoluble, para ello tomamos primitivas en
\[\frac{x'(t)}{x(t)} = -\lambda \implies \log(x) = -\lambda t + c
	\implies x = e^c e^{-\lambda t} = k e^{-\lambda t}\]
con lo que nos queda \(x = k e^{-\lambda t}\), llamamos a esta fórmula solución
general de la ecuación, ya que nos da todo el conjunto de soluciones variando
\(k\).

Observamos que el conjunto de soluciones en un espacio vectorial de dimensión 1,
esto se debe a que la ecuación es lineal y de dimensión 1. Esto intuitivamente
se debe a que hemos acotado el espacio vectorial de dimensión infinita de todas
las derivadas de la función a uno de dimensión finita.

Vemos ahora otro ejemplo de modelado más complicado, el movimiento del péndulo.
El punto de masa \(m\) es el peso del péndulo y llamaremos \(\theta\) al ángulo
formado entre la recta perpendicular al plano sobre el que esta apoyado el
péndulo y que parte de su punto de apoyo y la cuerda. Denotaremos la longitud
de la cuerda por \(l\) Recordamos que \(F = ma\) donde \(a\) es la aceleración. 
Tenemos entonces que
\[-m g \sin\theta = m l \theta''\]
por tanto
\[\theta'' = -\frac{g}{l} \sin\theta\]
puesto que \(\theta\) aparece dentro del seno, esta ecuación es no lineal y de
orden 2, puesto que aparece la derivada segunda de \(\theta\). En este caso 
\(f(t, \theta, \theta') = -\frac{g}{l} \sin\theta\) y \(\Omega = \R \times
(-\pi, \pi) \times \R\). Si \(\theta\) es suficientemente próximo a \(0\)
podemos aproximar \(\sin\theta\) a \(\theta\) con lo que tenemos la ecuación
lineal de segundo orden
\[\theta'' = -\frac{g}{l}\theta.\]

\end{document}
