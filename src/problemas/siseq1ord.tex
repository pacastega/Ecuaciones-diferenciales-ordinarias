\documentclass[../main.tex]{subfiles}

\begin{document}

\begin{problem}
	Resolver la ecuación diferencial asociada al modelo SIS:
	\[x' = \kappa x(N - x) - \beta x.\]

	Estudiar el problema del valor inicial asociado y obtener el teorema del
	umbral para este modelo.
\end{problem}

\begin{solution}
	Utilizamos el cambio de variable \(y = 1/x\), donde \(x > 0\), con lo que 
	\(y' = -x'/x^2\), sustituyendo
	\[y' = \kappa + (\beta - \kappa N) \frac{1}{x} 
		= \kappa + (\beta - \kappa N) y,\]
	obtenemos así una ecuación lineal, cuya solución general es
	\[y(t) = \frac{\kappa}{N\kappa - \beta} + x e^{(\beta - N\kappa)t},\]
	en la variable original \(x(t) = 1/y(t)\). Nos planteamos ahora el problema
	del valor inicial con \(x(t_0) = x_0\), sustituyendo en la solución obtenida
	\[x_0 = \frac{1}{
		\frac{\kappa}{N\kappa - \beta} + \parens{\frac{1}{x_0} 
		- \frac{\kappa}{N\kappa - \beta}} e^{(\beta - \kappa N)(t - t_0)}}.\]
	Nos preguntamos ahora cuál es el límite cuando \(t\) tiende a infinito.
	Tenemos dos casos, si \(\beta > \kappa N\) entonces 
	\(\lim_{t \to \infty} x(t) = 0\), por lo que la enfermedad desaparece.
	Si \(\beta < \kappa N\) el límite es \(N - \beta/\kappa\), por lo que la
	enfermedad se hace endémica. En el caso \(\beta = \kappa N\) no podemos
	sustituir en la solución por lo que consideramos \(y' = \kappa\) e
	\(y = \kappa t + c\), con esto \(x = 1/(\kappa t + c)\), sustituyendo en la
	condición inicial 
	\[x(t) = \frac{1}{\kappa(t - t_0) + \frac{1}{x_0}},\]
	con lo que la enfermedad también desaparece. La diferencia entre el caso uno
	y el tres es que en el primero la función tiende a \(0\) mucho más rápido
	(exponencialmente) que en el tercero.
\end{solution}

\begin{problem}
	Resolver los problemas lineal y cuadrático asociados al problema de caída
	libre con resistencia al aire. Es decir, la ecuación
	\[v' = g - \frac{\kappa}{m} v^n, \quad n \in \set{1, 2}.\]
\end{problem}

\end{document}
