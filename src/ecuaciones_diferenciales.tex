\documentclass{report}

\usepackage[a4paper]{geometry}
\usepackage[spanish]{babel}

\usepackage{subfiles} 
\usepackage{multicol}
\usepackage{hyperref}
\usepackage{parskip}
\usepackage{rotating}
\usepackage[toc]{appendix}
\usepackage{tikz}
\usetikzlibrary{arrows, babel}

\usepackage{general}
\usepackage{conjuntos}
\usepackage{teoremas}
\usepackage{calculo}
\usepackage{algebra}

\renewcommand\appendixpagename{Apéndices}
\renewcommand{\appendixtocname}{Apéndices}
\renewcommand{\theequation}{\thechapter.\arabic{equation}}

\def\Snospace~{\S{}}
\AtBeginDocument{\renewcommand*\sectionautorefname{\Snospace}}
\AtBeginDocument{\renewcommand*\subsectionautorefname{\Snospace}}

\newcommand{\verteq}{\rotatebox[origin=c]{90}{$\mkern1mu=$}}

\title{Elementos de ecuaciones diferenciales ordinarias}
\author{Ricardo Maurizio Paul \and Pablo Castellanos García}
\date{Curso 2020/2021}

\begin{document}
\maketitle
\tableofcontents

\setcounter{chapter}{-1}
\chapter{Introducción}
\subfile{teoria/intro}

\chapter{Ecuaciones diferenciales lineales de primer orden}
\label{chap:capitulo1}
\subfile{teoria/eqlin}

\chapter{Sistemas de ecuaciones diferenciales lineales de primer orden}
\label{chap:capitulo2}
\subfile{teoria/siseq1ord}


\chapter{Ecuación diferencial escalar de orden superior}
\label{chap:capitulo3}
\subfile{teoria/ecesordn.tex}

\chapter{Análisis cualitativo de los sistemas de ecuaciones diferenciales lineales}
\label{chap:capitulo3}
\subfile{teoria/cualitativo.tex}

\begin{appendices}

\chapter{Problemas resueltos}

\section{Problemas del capítulo~\ref{chap:capitulo1}}
\subfile{problemas/eqlin}

\section{Problemas del capítulo~\ref{chap:capitulo2}}
\subfile{problemas/siseq1ord}

\end{appendices}

\end{document}
