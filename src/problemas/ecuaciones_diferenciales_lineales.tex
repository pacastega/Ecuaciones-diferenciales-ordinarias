\documentclass[../main.tex]{subfiles}

\begin{document}

\subsection{Hoja 1}

\begin{problem}
	Encontrar la solución general de las siguientes ecuaciones diferenciales,
	indicando su intervalo de definición.

	\begin{multicols}{2}
	\begin{enumerate}[a)]
		\item \(\displaystyle x' + x \cos t = 0\)
		\item \(\displaystyle x' + x\sqrt{t} \sin t = 0\)
		\item \(\displaystyle x' + \frac{2t}{1 + t^2}x = \frac{1}{1 + t^2}\)
		\item \(\displaystyle x' + x = t e^t\)
	\end{enumerate}
	\end{multicols}
\end{problem}

\begin{solution}
	\begin{enumerate}[a), wide, labelwidth=0pt, labelindent=0pt]
		\item Se trata de una ecuación homogénea, puesto que 
			\(\cos t \in C(\R)\), el intervalo de definición de la ecuación es
			\(\R\). Separando variables:
			\begin{align*}
				\frac{x'}{x} = -\cos t &\iff 
				\log\abs{x(t)} = \int_0^t -\cos s \dif s + c 
				= -(\sin t - \sin 0) + c \\
				&\iff x(t) = k e^{-\sin t}.
			\end{align*}

		\item El intervalo de definición es \(\R^+ \cup \set{0}\), utilizando la
			fórmula obtenida en~\ref{sec:hom}:
			\[x(t) = k e^{\int^t \sqrt{s} \sin s \dif s},\]
			la integral no posee una primitiva elemental, por lo que no podemos
			simplificar más la solución.

		\item Tanto \(a(t)\) como \(b(t)\) son continuas en \(\R\). Utilizamos
			el método de variación de constantes, resolvemos primero la ecuación
			homogénea asociada:
			\[x_n(t) = k e^{\int_0^t \frac{-2s}{1 + s^2} \dif s},\]
			notamos que la derivada de \(-\log(1 + s^2)\) coincide con el
			interior de nuestra integral, por tanto:
			\[x_n(t) = k e^{-\log(1 + t^2)} = \frac{k}{1 + t^2}.\]

			Ahora, sea \(x_p = \frac{k(t)}{1 + t^2}\) solución de la ecuación
			diferencial, entonces
			\[x_p'(t) = k'(t)\frac{1}{1 + t^2} + k(t)\frac{-2t}{(1 + t^2)^2}\]
			sustituyendo en la ecuación obtenemos que
			\[x_p'(t) = \frac{-2t}{1 + t^2} x_p(t) + \frac{1}{1 + t^2}
				= k(t)\frac{-2t}{(1 + t^2)^2}  + \frac{1}{1 + t^2}.\]

			Igualando hallamos que \(k'(t) = 1\) y, por ende, \(k(t) = t\).
			La solución general es
			\[x(t) = x_n(t) + x_p(t) = \frac{k + t}{1 + t^2}.\]

		\item Tanto \(a = 1\) como \(b\) son continuas en todo \(\R\) por lo que
			este es nuestro intervalo de definición. Resolvemos la ecuación
			homogénea asociada, obteniendo \(x_n(t) = k e^{-t}\). Sea 
			\(x_p(t) = k(t) e^{-t}\) solución de la ecuación, entonces:
			\[\begin{cases}
				x_p'(t) = k'(t) e^{-t} - k(t) e^{-t} \\
				x_p'(t) = t e^t - x_p(t) = t e^t - k(t) e^{-t}
			  \end{cases}
			  \leadsto
			  k'(t) = t e^{2t}.
			\]
			Integrando por partes:
			\[k(t) = \int t e^{2t} \dif t 
				= \frac{t e^{2t}}{2} - \int \frac{e^{2t}}{2} \dif t
				= \frac{t e^{2t}}{2} - \frac{e^{2t}}{4}.\]
			La solución general de la ecuación es pues
			\[x(t) = x_n(t) + x_p(t) = (k + k(t))e^{-t}
				= k e^{-t} + \frac{t e^{t}}{2} - \frac{e^{t}}{4}.\]
	\end{enumerate}
\end{solution}

\end{document}
