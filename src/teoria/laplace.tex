\documentclass[../ecuaciones_diferenciales.tex]{subfiles}

\begin{document}
En este capítulo vamos a estudiar la transformada de Laplace y su utilidad a la
hora de resolver ecuaciones diferenciales.

La transformada de Laplace es un ejemplo de transformada integral (en
particular, lineal y continua). Para entender mejor este concepto, vamos a
hacer un recordatorio de las transformaciones lineales discretas:

Cualquier transformación lineal \(T : \R^n \to \R^m\) se corresponde con una
matriz \(A \in \mathcal{M}_{m \times n}\). Cualquier vector \(v \in \R^n\) se
puede ver como una aplicación
\(v : \{1, \dots, n\} \to \R,\ i \mapsto v(i) = v_i\) y, en particular, una
transformación lineal actuando sobre un vector, \(T(v)\), es una aplicación
\[T(v) : \{1, \dots, m\} \to \R, \qquad j \mapsto T(v)(j) = \sum_i A(j,i)v(i).\]
Una \emph{transformada integral} es el análogo continuo de esta idea. Es una
aplicación que lleva una función \(f\) a otra función \(\hat{f}\) mediante una
``fórmula'' del tipo
\[\hat{f}(s) = \int_0^\infty \kappa(s,t) f(t) \dif t,\]
donde \(\kappa\) recibe el nombre de \emph{kernel} o \emph{núcleo integral}.

La transformada de Laplace viene dada por el núcleo \(\kappa(s,t) =
e^{-st}\). La transformada de Fourier, estrechamente relacionada, viene dada por
\(\kappa(s,t) = e^{-ist}\).

\begin{remark}
	Si \(s \in \Complex\) en la transformada de Laplace, la parte imaginaria de
	\(\hat{f}\) se corresponde con la \(\hat{f}\) de la transformada de
	Fourier. Aquí nos centraremos en \(s \in \R\).
\end{remark}

Entrando en más detalle:

\begin{definition}
	Sea \(f(t)\) una función continua a trozos en \([0,+\infty)\). Su
	\emph{transformada de Laplace} es la función
	\[\L\{f(t)\} : s \mapsto \int_0^\infty f(t)e^{-st} \dif t.\]
	El \emph{dominio} de \(\L\{f(t)\}\) es el conjunto de \(s\) para los que la
	integral converge.
\end{definition}

\begin{remark}
	El dominio de \(\L\{f(t)\}\) depende de \(f(t)\).
\end{remark}

\begin{remark}
	En general, la transformada de Laplace se puede definir en contextos más
	generales, pero aquí nos basta con continuidad a trozos.
\end{remark}

\begin{remark}
	Se trata de un operador lineal, es decir, \(\L\{\alpha f(t) + \beta g(t)\} =
	\alpha \L\{f(t)\} + \beta \L\{g(t)\}\).
\end{remark}

\begin{notation}
	Denotaremos a las funciones en \(t\) con letras minúsculas y
	a sus transformadas con la correspondiente letra mayúscula. Por
	ejemplo, \(\L\{f(t)\}(s) = F(s)\) y \(\L\{g(t)\}(s) = G(s)\).
\end{notation}

\begin{notation}
	Usaremos \([\;\cdot\;]_{t=0}^{t=\infty}\) o simplemente
	\([\;\cdot\;]_0^\infty\) para referirnos a
	\(\lim \limits_{b \to \infty} [\;\cdot\;]_0^b\).
\end{notation}

\begin{notation}
	Frecuentemente escribiremos \(\L\{f(t)\}\) en lugar de
	\(\L\{f(t)\}(s)\) por simplicidad.
\end{notation}

\section{Transformadas de Laplace de funciones elementales}

\begin{itemize}
	\item \(f(t) = 1\):
	      \[\L\{1\} = \int_0^\infty e^{-st} \dif t = \frac{1}{s}
		      \left[-e^{-st}\right]_0^\infty = \frac{1}{s}(0-(-1)) =
		      \frac{1}{s},\]
	      si \(s > 0\), mientras que la integral diverge en caso contrario.
	\item \(f(t) = t\):
	      \[\L\{t\} = \int_0^\infty e^{-st}t \dif t =
		      \left[-\frac{te^{-st}}{s}\right]_0^\infty + \frac{1}{s} \int_0^\infty
		      e^{-st} \dif t = 0 + \frac{1}{s} \L\{1\} = \frac{1}{s^2},\]
	      si \(s > 0\), mientras que la integral diverge en caso contrario.
	\item \(f(t) = t^n,\ n \in \N\):
	      \[\L\{t^n\} = \frac{n!}{s^{n+1}} \quad (s>0).\]
	      Se prueba inmediatamente por inducción.
	\item \(f(t) = e^{at}\):
	      \[\L\{e^{at}\} = \int_0^\infty e^{at}e^{-st} \dif t = \int_0^\infty
		      e^{-(s-a)t} \dif t = \frac{1}{s-a}\left[-e^{-(s-a)t}\right]_0^\infty =
		      \frac{1}{s-a} \quad (s>a)\]
	\item \(f(t) = \sin kt\):
	      \[\L\{\sin kt\} = \int_0^\infty \sin kt e^{-st} \dif t = \cdots = \frac{k}{s^2} -
		      \frac{k^2}{s^2} \L\{\sin kt\} \implies \L\{\sin kt\} = \frac{k}{s^2+k^2}
		      \quad (s>0)\]
	\item \(f(t) = \cos kt\):
	      \[\L\{\cos kt\} = \int_0^\infty \cos kt e^{-st} \dif t = \cdots = \frac{1}{s} -
		      \frac{k^2}{s^2}\L\{\cos kt\} \implies  \L\{\cos kt\} = \frac{s}{s^2+k^2}
		      \quad (s>0)\]
\end{itemize}

Damos ahora una condición suficiente de existencia de la transformada de Laplace.

\begin{definition}
	\(f(t)\) es \emph{de orden exponencial} con constante \(c\) si existen \(M, T
	> 0\) tales que
	\[|f(t)| \leq Me^{ct} \quad \forall t \geq T.\]
\end{definition}

\begin{theorem}
	Si \(f(t)\) es continua a trozos y de orden exponencial con constante \(c\),
	entonces su transformada de Laplace \(F(s)\) existe si \(s > c\) (i.e. la
	integral converge). Además, en este caso, \(F(s) \to 0\) cuando \(s \to +\infty\).
	\begin{proof}
		Por definición,
		\[F(s) = \int_0^\infty f(t)e^{-st} \dif t,\]
		de donde
		\begin{align*}
			\lim_{b \to \infty} \int_0^b |f(t)|e^{-st} \dif t
			 & \leq \lim_{b \to \infty} \left( \int_0^T |f(t)|e^{-st} \dif t
			+ \int_T^b Me^{ct}e^{-st} \dif t \right)                         \\
			 & = \int_0^T |f(t)|e^{-st} \dif t + \lim_{b \to \infty}
			\int_T^b Me^{-(s-c)t} \dif t,
		\end{align*}
		y el último límite converge si \(s>c\). Para terminar,
		\(\lim_{s \to \infty} F(s) = 0\) sale del teorema de intercambiar
		límites e integrales.
	\end{proof}
\end{theorem}

\section{Transformada de Laplace inversa}
Si \(\L\{f(t)\} = F(s)\), diremos que \(f(t)\) es la \emph{transformada de
	Laplace inversa} de \(F(s)\) y lo denotaremos como \(f(t) = \L^{-1}\{F(s)\}\).

\begin{remark}
	Si \(f(t)\) es continua, la transformada inversa está bien definida
	(i.e. \(f(t)\) es la única función continua tal que su transformada es \(F(s)\)).
\end{remark}

\begin{remark}
	\(\L^{-1}\) también es lineal.
\end{remark}

\begin{example}
	Hallar la transformada de Laplace inversa de
	\[F(s) = \frac{-2s+6}{s^2+4}.\]
	Usamos la linealidad del operador \(\L^{-1}\), para lo que resulta útil
	descomponer \(F(s)\) en funciones cuya transformada inversa ya conozcamos:
	\[\L^{-1}\left\{\frac{-2s+6}{s^2+4}\right\} =
		-2\L^{-1}\left\{\frac{s}{s^2+2^2}\right\} +
		\frac{6}{2}\L^{-1}\left\{\frac{2}{s^2+2^2}\right\} = -2 \cos 2t + 3 \sin 2t.\]
\end{example}

\section{Transformada de Laplace y derivadas}
A la hora de aplicar la transformada de Laplace para resolver ecuaciones
diferenciales, es imprescindible saber cómo se comporta respecto a las derivadas
sucesivas de una función. El siguiente teorema es fundamental para nuestro
objetivo:

\begin{theorem}
	Si \(f, f', \dots, f^{(n-1)}\) son continuas en \([0,+\infty)\) y de orden
	exponencial con constante \(c\) y \(f^{(n)}\) es continua \emph{a trozos} en
	\([0,+\infty)\) y de orden exponencial con constante \(c\), entonces
	\[\L\left\{f^{(n)}(t)\right\} = s^nF(s) - s^{n-1}f(0) - s^{n-2}f'(0) - \cdots -
		s^0f^{(n-1)}(0) \qquad (s>c)\]
	\begin{proof}
		Procedemos por inducción sobre el orden de la derivada. Para el caso base
		\(n=1\) desarrollamos la definición mediante integración por partes:
		\begin{align*}
			\L\{f'(t)\} & = \int_0^\infty f'(t)e^{-st} \dif t =
			\left[f(t)e^{-st}\right]_0^\infty - (-s) \int_0^\infty f(t)e^{-st} \dif t \\
			            & = (0-f(0)) + sF(s) = sF(s) - f(0) \qquad (s > c)
		\end{align*}
		Suponiendo el resultado cierto para \(n-1\), vamos a probar que se cumple
		para \(n\):
		\begin{align*}
			\L\left\{f^{(n)}(t)\right\}
			 & = \L\left\{\left(f^{(n-1)}\right)'(t)\right\}
			= s\L\left\{f^{(n-1)}(t)\right\} - f^{(n-1)}(0)        \\
			 & = s \left( s^{n-1}F(s) - s^{n-2}f(0) - s^{n-3}f'(0)
			- \cdots - s^0f^{(n-2)}(0) \right) - f^{(n-1)}(0)      \\
			 & = s^nF(s) - s^{n-1}f(0) - s^{n-2}f'(0) - \cdots
			- s^1f^{(n-2)}(0) - s^0f^{(n-1)}(0),
		\end{align*}
		con lo que queda probado el teorema.
	\end{proof}
\end{theorem}

Después de este resultado, ya estamos listos para ver cómo se puede aplicar la
transformada de Laplace para resolver ecuaciones diferenciales:

\begin{example}
	Resolver el problema de valor inicial
	\[
		\begin{cases}
			y' + 3y = 13 \sin 2t \\
			y(0) = 6
		\end{cases}
	\]
	Vamos a hacerlo en tres pasos:
	\begin{enumerate}[1)]
		\item Aplicamos la transformada de Laplace a ambos lados de la ecuación
		      diferencial y simplificamos aprovechando la linealidad:
		      \[\L\{y'\} + 3\L\{y\} = 13\L\{\sin 2t\}\]
		\item Operamos, teniendo en cuenta que
		      \begin{gather*}
			      \L\{y'\} = sY(s) - y(0) = sY(s) -  6, \\
			      \L\{\sin 2t\} = \frac{2}{s^2+4}.
		      \end{gather*}
		      Así, la ecuación transformada es
		      \[(sY(s) - 6) + 3Y(s) = \frac{26}{s^2+4}\]
		      o, despejando \(Y(s)\),
		      \[
			      Y(s) = \frac{6}{s+3} + \frac{26}{(s+3)(s^2+4)} = \frac{6}{s+3} + \left(
			      \frac{2}{s+3} + \frac{-2s+6}{s^2+4} \right) = \frac{8}{s+3} + \frac{-2s+6}{s^2+4}
		      \]
		\item Aplicamos la transformada inversa:
		      \begin{align*}
			      y(t) & = 8 \L^{-1}\left\{\frac{1}{s+3}\right\} + 3
			      \L^{-1}\left\{\frac{2}{s^2+2^2}\right\} -2
			      \L^{-1}\left\{\frac{s}{s^2+2^2}\right\}            \\
			           & = 8e^{-3t} + 3\sin 2t - 2\cos 2t.
		      \end{align*}
	\end{enumerate}
\end{example}

Como se aprecia en el ejemplo, la utilidad de la transformada de Laplace consiste
en que transforma ecuaciones diferenciales en ecuaciones algebraicas, que son
mucho más fáciles de resolver. Veamos otro ejemplo un poco más complicado:

\begin{example}
	Resolver el PVI
	\[
		\begin{cases}
			y'' - 3y' + 2y = e^{-4t} \\
			y(0) = 1, y'(0) = 5
		\end{cases}
	\]
	Seguimos los mismos pasos que antes:
	\begin{enumerate}[1)]
		\item Aplicamos \(\L\):
		      \[\L\{y''\} - 3\L\{y'\} + 2\L\{y\} = \L\{e^{-4t}\}\]
		\item Operamos, teniendo en cuenta que
		      \begin{gather*}
			      \L\{y''\} = s^2Y(s) - sy(0) - y'(0) = s^2Y(s) - s - 5, \\
			      \L\{y'\} = sY(s) - y(0) = sY(s) - 1,
		      \end{gather*}
		      con lo que la ecuación queda
		      \begin{gather*}
			      \left(s^2Y(s) - s - 5\right) - 3\left(sY(s) - 1\right) + 2Y(s) =
			      \frac{1}{s+4} \implies \\
			      Y(s) = \frac{s+2}{s^2-3s+2} + \frac{1}{(s^2-3s+2)(s+4)} =
			      \frac{-16/5}{s-1} + \frac{25/6}{s-2} + \frac{1/30}{s+4}.
		      \end{gather*}
		\item Aplicamos \(\L^{-1}\):
		      \begin{align*}
			      y(t) & = \L^{-1}\{Y(s)\}
			      = -\frac{16}{5}\L^{-1}\left\{\frac{1}{s-1}\right\}
			      + \frac{25}{6}\L^{-1}\left\{\frac{1}{s-2}\right\}
			      + \frac{1}{30}\L^{-1}\left\{\frac{1}{s+4}\right\} \\
			           & = -\frac{16}{5}e^t + \frac{25}{6}e^{2t}
			      + \frac{1}{30}e^{-4t}.
		      \end{align*}
	\end{enumerate}
\end{example}

\begin{remark}
	La transformada de Laplace es especialmente útil para problemas de valor
	inicial, aunque también sirve para encontrar la solución general de una
	ecuación diferencial. Si al aplicar el operador \(\L\) a las derivadas de la
	función no se sustituyen directamente los valores
	\(x(0), x'(0), \dots, x^{(n-1)}(0)\), el resultado final queda en función de
	estos valores, que son precisamente los parámetros que permiten generar todo
	el espacio de soluciones.
\end{remark}

\section{Propiedades de \(\L\) y \(\L^{-1}\)}
\begin{proposition}
	Si \(F(s)\) es la transformada de Laplace de \(f(t)\), entonces
	\[\L\left\{e^{at}f(t)\right\} = F(s-a).\]
	\begin{proof}
		Basta aplicar la definición:
		\[\L\left\{e^{at}f(t)\right\}(s) = \int_0^\infty e^{at}f(t) e^{-st} \dif t
			= \int_0^\infty f(t) e^{-(s-a)t} \dif t = \L\{f(t)\}(s-a)\]
	\end{proof}
\end{proposition}

\begin{example}
	Calcular la transformada de Laplace de \(e^{-2t} \cos 4t\).
	\[\L\left\{e^{-2t} \cos 4t\right\}(s) = \L\{\cos 4t\}(s+2)
		= \frac{(s+2)}{(s+2)^2+16}\]
\end{example}

Normalmente este resultado se usa en sentido inverso: a menudo queremos
encontrar transformadas inversas de funciones que son traslación de
transformadas que conocemos.

\begin{example}
	Calcular la transformada inversa de \((s-3)^{-2}\).
	\[\L^{-1}\left\{\frac{1}{(s-3)^2}\right\} = \L^{-1}\left\{\restr{\frac{1}{s^2}}{s
			\to s-3}\right\} = te^{3t},\]
	teniendo en cuenta que
	\[\L\{t\} = \frac{1}{s^2}.\]
\end{example}

\begin{notation}
	Cuando queramos decir explícitamente que una función es
	traslación de otra, lo denotaremos como en el ejemplo, es decir,
	\[\restr{f(x)}{x \to x+a} = f(x+a).\]
\end{notation}

También es útil hacer traslaciones en \(t\), para lo que introducimos el
siguiente resultado.

\begin{definition}
	Se define la \emph{función escalón} o \emph{función de Heaviside} \(\mathcal{U}\) como
	\[\mathcal{U}(t) =
		\begin{cases}
			0 & \mathrm{si}\ t < 0    \\
			1 & \mathrm{si}\ t \geq 0
		\end{cases}
	\]
	Aquí nos interesará considerar traslaciones de \(\mathcal{U}\) restringidas a
	\([0,+\infty)\):
	\[\mathcal{U}(t-a) =
		\begin{cases}
			0 & \mathrm{si}\ 0 \leq t < a \\
			1 & \mathrm{si}\ t \geq a
		\end{cases}
	\]
\end{definition}

La función de Heaviside permite definir fácilmente funciones a trozos.

\begin{example}
	Dada la función \(f\) definida como
	\[f(t) =
		\begin{cases}
			g(t) & \mathrm{si}\ 0 \leq t < a \\
			h(t) & \mathrm{si}\ t \geq a
		\end{cases}
	\]
	se puede expresar de forma más concisa como
	\(f(t) = g(t) + (h(t)-g(t))\mathcal{U}(t-a)\).
\end{example}

\begin{proposition}
	Si \(F(s)\) es la transformada de Laplace de \(f(t)\), entonces
	\[\L\{f(t-a)\mathcal{U}(t-a)\} = e^{-as}F(s).\]
	\begin{proof}
		Basta aplicar la definición:
		\begin{align*}
			\L\{f(t-a)\mathcal{U}(t-a)\}
			 & = \int_0^\infty f(t-a)\mathcal{U}(t-a) e^{-st} \dif t
			= \int_a^\infty f(t-a)e^{-st} \dif t                     \\
			 & = \int_0^\infty f(u)e^{-s(u+a)} \dif u
			= e^{-sa} \int_0^\infty f(u) e^{-su} \dif u              \\
			 & = e^{-sa}F(s),
		\end{align*}
		haciendo el cambio de variable \(u = t - a\).
	\end{proof}
\end{proposition}

\begin{example}
	Hallar la transformada inversa de \(\frac{1}{s-4}e^{-2s}\).

	Teniendo en cuenta que
	\[\L^{-1}\left\{\frac{1}{s-4}\right\} = e^{4t},\]
	aplicando la proposición anterior con \(a=2\) se tiene
	\[\L^{-1}\left\{\frac{1}{s-4}e^{-2s}\right\} = e^{4(t-2)}\mathcal{U}(t-2).\]
\end{example}

\begin{theorem}
	Si \(n \in \N\) y \(f(t)\) es continua a trozos y de orden exponencial, entonces
	\[(-1)^n\L\left\{t^nf(t)\right\} = \frac{\dif {}^n}{\dif s^n} F(s).\]
	\begin{proof}
		Partiendo de la definición
		\[F(s) = \int_0^\infty f(t)e^{-st} \dif t,\]
		basta aplicar el teorema de derivación bajo el signo integral para obtener
		\[\frac{\dif F(s)}{\dif s} = \int_0^\infty (-t)f(t)e^{-st} \dif t =
			-\L\{tf(t)\}.\]
		El caso general se demuestra igual por inducción.
	\end{proof}
\end{theorem}

\begin{definition}
	Si \(f\) y \(g\) son continuas a trozos en \([0,+\infty)\), se define su
	\emph{convolución} como
	\[(f \ast g)(t) = \int_0^t f(u) g(t-u) \dif u.\]
\end{definition}

\begin{remark}
	\(f \ast g = g \ast f.\)
	\begin{proof}
		Es inmediato haciendo el cambio de variable \(u = t-v\) en la integral:
		\[(f \ast g)(t) = \int_0^t f(u) g(t-u) \dif u = \int_t^0 f(t-v)g(v)
			(-\dif v) = \int_0^t g(v) f(t-v) \dif v = (g \ast f)(t).\]
	\end{proof}
\end{remark}

\begin{theorem}
	Si \(f\) y \(g\) son continuas a trozos en \([0,\infty)\) y de orden
	exponencial, y si \(F\) y \(G\) son sus transformadas de Laplace, entonces
	\[\L\{f \ast g\} = F(s) \cdot G(s).\]
	\begin{proof}
		Por definición,
		\[\L\{f \ast g\} = \int_0^\infty \left( \int_0^t f(u)g(t-u) \dif u \right)
			e^{-st} \dif t.\]
		Observamos que estamos integrando sobre la región \(R = \{t > u > 0\} \subseteq
		\R^2\); aplicando el teorema de Fubini:
		\begin{align*}
			\L\{f \ast g\}
			 & = \iint_R f(u) g(t-u) \dif u \dif t
			= \int_0^\infty \int_u^\infty f(u)g(t-u)e^{-st} \dif t \dif u            \\
			 & = \int_0^\infty f(u) \left(\int_u^\infty g(t-u)e^{-st} \dif t \right)
			\dif u = \int_0^\infty f(u) \left( \int_0^\infty g(v)e^{-s(v+u)} \dif v
			\right) \dif u,
		\end{align*}
		donde hemos hecho el cambio de variable \(v = t-u\) en la integral
		interior. Desarrollando, esto se puede separar en dos integrales:
		\[\int_0^\infty f(u) e^{-su} \left( \int_0^\infty g(v)e^{-sv} \dif v \right) \dif u
			= \left( \int_0^\infty f(u)e^{-su} \dif u \right)
			\left( \int_0^\infty g(v)e^{-sv} \dif v \right)
			= F(s) G(s),\]
		como queríamos.
	\end{proof}
\end{theorem}

A modo de resumen, incluimos en una tabla los resultados más relevantes del
capítulo:

\begin{table}[ht]
	\centering
	\begin{tabular}{c|c}
		Transformada                                                 & Comentarios
		\\[0.7em] \hline \\[-1.0em]
		\(\displaystyle \L\{1\} = \frac{1}{s}\)                      & \(s>0\)
		\\[0.7em] \hline \\[-1.0em]
		\(\displaystyle \L\{t^n\} = \frac{n!}{s^{n+1}}\)             & \(s>0\)
		\\[0.7em] \hline \\[-1.0em]
		\(\displaystyle \L\{e^{at}\} = \frac{1}{s-a}\)               & \(s>a\)
		\\[0.7em] \hline \\[-1.0em]
		\(\displaystyle \L\{\sin kt\} = \frac{k}{s^2+k^2}\)          & \(s>0\)
		\\[0.7em] \hline \\[-1.0em]
		\(\displaystyle \L\{\cos kt\} = \frac{s}{s^2+k^2}\)          & \(s>0\)
		\\[0.7em] \hline \\[-1.0em]
		\(\displaystyle \L\{f^{(n)}(t)\} = s^nF(s) - s^{n-1}f(0) - \cdots -
		f^{(n-1)}(0)\)                                               & ---
		\\[0.4em] \hline \\[-1.0em]
		\(\displaystyle \L\left\{e^{at}f(t)\right\} = F(s-a)\)       & ---
		\\[0.4em] \hline \\[-1.0em]
		\(\displaystyle \L\{f(t-a)\mathcal{U}(t-a)\} = e^{-as}F(s)\) & ---
		\\[0.4em] \hline \\[-1.0em]
		\(\displaystyle (-1)^n\L\left\{t^nf(t)\right\} = \frac{\dif {}^n}{\dif s^n}
		F(s)\)                                                       & ---
		\\[0.7em] \hline \\[-1.0em]
		\(\displaystyle \L\{f \ast g\} = F(s) \cdot G(s)\)           & ---
	\end{tabular}%
	\caption{Transformadas de Laplace de algunas funciones importantes}
\end{table}

\end{document}
