\documentclass[../main.tex]{subfiles}

\begin{document}

\begin{problem}
	Resolver la ecuación diferencial asociada al modelo SIS:
	\[x' = \kappa x(N - x) - \beta x.\]

	Estudiar el problema del valor inicial asociado y obtener el teorema del
	umbral para este modelo.
\end{problem}

\begin{solution}
	Utilizamos el cambio de variable \(y = 1/x\), donde \(x > 0\), con lo que 
	\(y' = -x'/x^2\), sustituyendo
	\[y' = \kappa + (\beta - \kappa N) \frac{1}{x} 
		= \kappa + (\beta - \kappa N) y,\]
	obtenemos así una ecuación lineal, cuya solución general es
	\[y(t) = \frac{\kappa}{N\kappa - \beta} + x e^{(\beta - N\kappa)t},\]
	en la variable original \(x(t) = 1/y(t)\). Nos planteamos ahora el problema
	del valor inicial con \(x(t_0) = x_0\), sustituyendo en la solución obtenida
	\[x_0 = \frac{1}{
		\frac{\kappa}{N\kappa - \beta} + \parens{\frac{1}{x_0} 
		- \frac{\kappa}{N\kappa - \beta}} e^{(\beta - \kappa N)(t - t_0)}}.\]
	Nos preguntamos ahora cuál es el límite cuando \(t\) tiende a infinito.
	Tenemos dos casos, si \(\beta > \kappa N\) entonces 
	\(\lim_{t \to \infty} x(t) = 0\), por lo que la enfermedad desaparece.
	Si \(\beta < \kappa N\) el límite es \(N - \beta/\kappa\), por lo que la
	enfermedad se hace endémica. En el caso \(\beta = \kappa N\) no podemos
	sustituir en la solución por lo que consideramos \(y' = \kappa\) e
	\(y = \kappa t + c\), con esto \(x = 1/(\kappa t + c)\), sustituyendo en la
	condición inicial 
	\[x(t) = \frac{1}{\kappa(t - t_0) + \frac{1}{x_0}},\]
	con lo que la enfermedad también desaparece. La diferencia entre el caso uno
	y el tres es que en el primero la función tiende a \(0\) mucho más rápido
	(exponencialmente) que en el tercero.
\end{solution}

\begin{problem}
	Resolver los problemas lineal y cuadrático asociados al problema de caída
	libre con resistencia al aire. Es decir, la ecuación
	\[v' = g - \frac{\kappa}{m} v^n, \quad n \in \set{1, 2}.\]
\end{problem}

\begin{solution}
	Esta ecuación es separable, puesto que se puede expresar como
	\[\frac{v'}{g - \frac{\kappa}{m} v^n} = 1.\]
	Para resolverla así, distinguimos casos en función del valor de \(n\):
	\begin{itemize}
	  \item Caso \(n = 1\):
		Como las dos primitivas son elementales, integrando a ambos lados
		obtenemos sin mayor dificultad
		\[-\frac{m}{\kappa} \log \abs{g - \frac{\kappa}{m} v} = t + c_1\]
		y ahora sin más que despejar \(v\) llegamos a la solución general:
		\[v(t) = \frac{gm}{\kappa} - ce^{-\frac{\kappa}{m}t}.\]

		Observamos que \(\lim_{t \to \infty} v(t) = \frac{gm}{\kappa}\), es
		decir, la velocidad aumenta (o disminuye, en función del valor inicial)
		hasta una \emph{velocidad terminal}; esto contrasta con lo que ocurre en
		un sistema sin rozamiento, en el que la velocidad aumenta sin
		restricción.
	  \item Caso \(n = 2\): Para hallar una primitiva del término de la
		izquierda, lo separamos primero en fracciones simples; denotaremos
		\(\alpha := \frac{\kappa}{m}\) para simplificar la notación.
		\[\frac{1}{g - \alpha x^2} = \frac{\frac{1}{2\sqrt{g}}}{\sqrt{g} +
				\sqrt{\alpha}x} + \frac{\frac{1}{2\sqrt{g}}}{\sqrt{g} -
				\sqrt{\alpha}x}.\]
		  Así es inmediato hallar las primitivas:
		  \[t + c_1 = \frac{1}{2\sqrt{g}} \int \frac{v'(t)}{\sqrt{g} +
				\sqrt{\alpha}v(t)} + \frac{v'(t)}{\sqrt{g} - \sqrt{\alpha}v(t)}
			  \dif t = \frac{1}{2\sqrt{g\alpha}} \log \abs{\frac{\sqrt{g} +
				  \sqrt{\alpha}v}{\sqrt{g} - \sqrt{\alpha}v}}\]
		  y ahora basta despejar \(v\):
		  \[v =
			  \frac{\sqrt{g}}{\sqrt{\alpha}}\left(\frac{e^{2\sqrt{g\alpha}t+c_2}-1}{e^{2\sqrt{g\alpha}t+c_2}+1}\right)
			  =
			  \sqrt{\frac{gm}{\kappa}}\tanh\left(\sqrt{\frac{g\kappa}{m}}t+c\right).\]
		  Igual que antes, la velocidad se acaba estabilizando, aunque en este
		  caso el límite es \(\sqrt{\frac{gm}{\kappa}}\).
	\end{itemize}
\end{solution}

\end{document}
