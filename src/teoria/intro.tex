\documentclass[../ecuaciones_diferenciales.tex]{subfiles}

\begin{document}

\section{Primeras definiciones}

\begin{definition}[Ecuación diferencial]
	Una ecuación diferencial es una ecuación de la forma
	\[F(t, x, x', x'', \dots, x^{(n)}) = 0\]
	donde \(F : \Omega \to \R\) con \(\Omega \subset \R^{n + 2}\) abierto.
\end{definition}

\begin{definition}
	Se dice que \(x(t)\) es solución de una ecuación diferencial en el intervalo
	\(I \subset \R\) si \(x : I \to \R\) y
	\[F(t, x(t), x'(t), x''(t), \dots, x^{(n)}(t)) = 0, \quad \forall t \in I.\]
	Se cumple además que:
	\begin{enumerate}[i)]
		\item La función \(x\) tiene todas las derivadas hasta orden \(n\) y son
		      continuas.

		\item El punto \((t, x(t), x'(t), \dots, x^{(n)}(t)) \in \Omega\) para
		      todo \(t\) en \(I\).
	\end{enumerate}
\end{definition}

Supondremos que la ecuación diferencial está dada en forma normal.

\begin{definition}[Forma normal]
	Una ecuación diferencial está escrita en forma normal si es una expresión 
	de la forma
	\begin{equation} \label{eq:forma_normal}
		x^{(n)} = f(t, x', x'', \dots, x^{(n - 1)})
	\end{equation}
\end{definition}

Llamaremos \(\Omega \subset \R^{n + 1}\) al dominio de \(f\), esto se puede hacer
localmente sin pérdida de generalidad gracias al teorema de la función
implícita.

\begin{definition}[Orden]
	Llamaremos orden de la ecuación diferencial al número \(n\) en
	la ecuación dada en forma normal~\ref{eq:forma_normal}.
\end{definition}

\begin{definition}[Ecuación diferencial lineal]
	Una ecuación diferencial es lineal si \(f\) es de la forma
	\[f(t) = a_{n - 1}(t)x^{(n - 1)} + a_{n - 2}(t)x^{(n - 2)} +
		\dots + a_1(t)x' + a_0(t)x + b(t)\]
\end{definition}

\section{Ejemplos de modelado}

Antes de adentrarnos en el estudio riguroso de las ecuaciones diferenciales
presentamos algunos ejemplos que muestran como aparecen naturalmente al plantear
ciertos problemas físicos y químicos. En todos los casos comenzaremos
describiendo matemáticamente el sistema que estamos analizando, esto es creando
un modelo, que después podremos resolver utilizando técnicas que veremos en los
sucesivos capítulos.

\subsection{Desintegración de sustancias radiactivas}

Nos interesa hallar la masa de un cierto elemento radiactivo en un tiempo \(t\),
para ello escribimos \(x(t)\) como la masa en función del tiempo, con lo que
queremos despejar \(x\).

La clave que nos permite modelar este fenómeno es que la probabilidad de 
desintegración \(\lambda\) de un átomo es la misma para todos los átomos de la 
muestra, independientemente de la masa del material.
Nos interesa la masa en función del tiempo relativa, esto es el cambio en la
masa del material en un cierto tiempo
\[\frac{\Delta x}{x \Delta t} = \frac{x(t + \Delta t) - x(t)}{x(t) \Delta t}
	= -\lambda, \quad \lambda > 0,\] 
donde la última igualdad se deduce de lo dicho anteriormente. 
Llamaremos al término \(\frac{\Delta x}{x\Delta t}\) la tasa de
(de)crecimiento media. Tomando el límite \(\Delta t \to 0\) nos queda
\(x'(t)/x(t) = -\lambda\) para todo \(t\). Esto nos da una ecuación diferencial
lineal de primer orden (de coeficientes constantes).

En este caso, \(x\) es una masa, por lo que no puede ser negativa, por tanto
\[f(t, x) = -\lambda x\]
con \(f : \Omega \to \R\) y \(\Omega = \R \times (0, +\infty)\).

Esta ecuación diferencial es resoluble, para ello tomamos primitivas en
\[\frac{x'(t)}{x(t)} = -\lambda \implies \log(x) = -\lambda t + c
	\implies x = e^c e^{-\lambda t} = k e^{-\lambda t}\] con lo que nos queda
\(x = k e^{-\lambda t}\). Llamamos a esta familia de funciones \emph{solución general}
de la ecuación, y nos da todo el conjunto de soluciones variando \(k\). En
particular, el conjunto de soluciones es un subespacio vectorial de dimensión
uno del espacio vectorial \(C^1(\R)\), que tiene dimensión infinita.

\subsection{Movimiento armónico simple} 

Vemos ahora un ejemplo de modelado más complicado, el movimiento del péndulo.
Queremos hallar el ángulo del péndulo en un momento determinado, para ello
hacemos algunas observaciones. Llamaremos \(\theta\) al ángulo
formado entre la recta perpendicular al plano sobre el que esta apoyado el
péndulo y que parte de su punto de apoyo y la cuerda y \(m\) al punto de masa
donde se concentra todo el peso del mismo, por último denotaremos la longitud
de la cuerda por \(l\). 

Recordamos que, por la segunda ley de Newton, \(F = ma\) donde \(a\) es la 
aceleración. Tenemos entonces que
\[-m g \sin\theta = m l \theta''\]
por tanto
\[\theta'' = -\frac{g}{l} \sin\theta\]
puesto que \(\theta\) aparece dentro del seno, esta ecuación es no lineal y de
orden 2, ya que aparece la derivada segunda de \(\theta\). En este caso
\(f(t, \theta, \theta') = -\frac{g}{l} \sin\theta\) y \(\Omega = \R \times
(-\pi, \pi) \times \R\). 

Si \(\theta\) es suficientemente próximo a \(0\) podemos aproximar 
\(\sin\theta\) a \(\theta\) con lo que tenemos la ecuación
lineal de segundo orden
\[\theta'' = -\frac{g}{l}\theta.\]

Esta es la ecuación del movimiento armónico simple. Su solución general, que,
recordamos, es el conjunto de todas las soluciones es
\[\theta(t) = C_1 \sin \omega t + C_2 \cos \omega t\]
donde \(\omega^2 = g/l\), demostraremos esto más adelante. Este conjunto de
soluciones forma un espacio vectorial de dimensión 2. Demostraremos también que
el conjunto de soluciones de una ecuación lineal de orden \(n\) es un espacio
vectorial, o afín si existe un término independiente, de orden \(n\).

\subsection{Problema de valor inicial (PVI)}

En los ejemplos anteriores hemos obtenido soluciones generales para los
problemas propuestos, pero si conocemos el valor de \(x\) y sus derivadas en un
tiempo determinado la solución de la ecuación diferencial está
unívocamente determinada, como demostraremos.

\begin{definition}[Problema de valor inicial]
	Un problema de valor inicial es una ecuación diferencial
	\[x^{(n)} = f(t, x, x', \dots, x^{(n - 1)}),\]
	junto a su valor y el de sus derivadas en un tiempo determinado \(t_0\)
	\[x(t_0) = x_0,\ x'(t_0) = x_1,\ \dots\ ,\ x^{(n - 1)}(t_0) = x_{n - 1}.\]
\end{definition}

\begin{definition}[Condición inicial]
	Llamaremos condición inicial al valor de \(x\) y sus derivadas en un tiempo
	determinado \(t_0\) en un problema de valor inicial.
\end{definition}

\subsection{Datación por radiocarbono}

Veremos como es posible datar un objeto conociendo solo la cantidad carbono 
presente en el mismo.

Existen diversos isótopos del carbono, entre ellos el carbono-12, el cual es
estable, y el carbono-14, el isótopo radiactivo más común. El carbono-14, al ser
radiactivo debería desintegrarse y disminuir su concentración en la atmósfera,
pero debido a los rayos cósmicos que constantemente alcanzan la tierra la
proporción entre estos dos isótopos se mantiene estable en la atmósfera. A
través de la fotosíntesis estos isótopos radiactivos pasan a formar parte de las
plantas y después, mediante la cadena alimenticia, del resto de seres vivos.
Cuando un ser vivo muere la cantidad de carbono-12 contenida en el mismo se
mantiene constante, mientras que la de carbono-14 disminuye anualmente,
a un ritmo que depende de su probabilidad de desintegración
\(\lambda \sim \log 2 / 5730\).

Con esto, analizando la proporción entre carbono-12 y 14 en
material orgánico (madera, hueso, etc.) y comparándolo con una muestra viva es 
posible determinar la edad del resto. La creación de este método de
datación valió al químico estadounidense Willard Libby el premio Nobel de
química en
1960\footnote{\url{https://es.wikipedia.org/wiki/Datación_por_radiocarbono}}.
Veamos algunos ejemplos concretos de como se puede aplicar esta
técnica.

\begin{example}
	Si la cantidad de carbono-14 en un microorganismo es de \(10^{-6}\) gramos
	¿Qué cantidad habrá 3000 años después?
\end{example}

\begin{solution}
	Se entiende que el ciclo de vida del organismo
	es despreciable respecto a la edad que debemos datar. Tenemos que
	\(x' = -\lambda x\), a tiempo \(0\), cuando el organismo muere, tenemos que
	\(x(0) = 10^{-6}\). Se trata de un problema de valor inicial, ahora solo tenemos
	que encontrar \(x(3000)\). Empezamos resolviendo la ecuación diferencial
	\[x(t) = k e^{-\lambda t} = k e^{-\frac{\log 2}{5730} t},\]

	por lo que hemos visto antes \(k = x(0) = 10^{-6}\) con lo que
	\[x(t) = 10^{-6} e^{-\frac{\log 2}{5730} t}.\]

	Podemos obtener ahora el resultado \(x(3000) = 6,95 \cdot 10^{-7}\) gramos.
\end{solution}

Normalmente no sabemos la cantidad total de carbono presente en un organismo
antes de su muerte, ni el tiempo transcurrido desde entonces. Veamos ahora un
ejemplo más cercano a la realidad.

\begin{example}
	En una excavación en Nippur, Babilonia, en 1950 el carbón vegetal de la viga
	de un techo dio \(4,09\) desintegraciones por minuto y gramo, mientras que
	la madera viva da \(6,68\) desintegraciones, también por minuto y gramo.
	Nos preguntamos en qué año se construyó en edificio.
\end{example}

\begin{solution}
	Llamaremos \(R(t)\) a la tasa de desintegración total
	\[R(t) = \lambda x(t) = \lambda x(0) e^{-\lambda t},\]
	donde \(x(0)\) es la
	cantidad de carbono-14 en el trozo de madera cuando fue cortada y \(t\) el
	tiempo transcurrido desde que se corta el árbol hasta 1950. Buscamos entonces
	el año de construcción, para ello observamos que \(R(0) = \lambda x(0)\), de
	donde el cociente \(\frac{R(t)}{R(0)} = e^{-\lambda t}\) no depende de la
	cantidad de carbono-14 presente en el organismo vivo \(x_0\). Sustituyendo
	los datos y despejando
	\[\frac{4,09}{6,68} = \frac{R(t)}{R(0)} = e^{-\frac{\log 2}{5730} t}
		\implies t \approx 4055,\]
	medido en años. Así, concluimos que el edificio se construyó en el \(2105\)
	a.C.
\end{solution}

\end{document}
