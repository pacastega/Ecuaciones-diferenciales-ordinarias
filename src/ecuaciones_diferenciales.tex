\documentclass[a4paper, 10pt]{book}
%\documentclass[draft, a4paper, 10pt]{book}

\usepackage{geometry}
\usepackage[spanish]{babel}

\usepackage{subfiles} 
\usepackage{multicol}
\usepackage{hyperref}
\usepackage{parskip}
\usepackage{rotating}
\usepackage[toc]{appendix}
\usepackage{tikz}
\usetikzlibrary{arrows, babel}
\usepackage{subcaption}

\usepackage{general}
\usepackage{conjuntos}
\usepackage{teoremas}
\usepackage{calculo}
\usepackage{algebra}

\renewcommand{\L}{\mathcal{L}}

\renewcommand\appendixpagename{Apéndices}
\renewcommand{\appendixtocname}{Apéndices}
\renewcommand{\theequation}{\thechapter.\arabic{equation}}

\def\Snospace~{\S{}}
\AtBeginDocument{\renewcommand*\sectionautorefname{\Snospace}}
\AtBeginDocument{\renewcommand*\subsectionautorefname{\Snospace}}

\newcommand{\verteq}{\rotatebox[origin=c]{90}{$\mkern1mu=$}}

\title{Elementos de ecuaciones diferenciales ordinarias}
\author{Ricardo Maurizio Paul \and Pablo Castellanos García}
\date{Curso 2020--2021}

\begin{document}

\frontmatter

\maketitle

%\clearpage
\cleardoublepage
\vspace*{\fill}
\thispagestyle{empty} % Sin número de página
\begin{quotation}
\raggedleft
\em % énfasis
Duc, o parens celsique dominator poli, \\
Quocumque placuit; nulla parendi mora est. \\
Adsum inpiger. Fac nolle, comitabor gemens \\
Maiusque patiar, facere quod licuit bono. \\
Ducunt volentem fata, nolentem trahunt. \\
\bigskip
--- Seneca
\end{quotation}
\vspace*{\fill}

\tableofcontents

\chapter{Introducción}
\subfile{teoria/intro}

\mainmatter

\chapter{Ecuaciones lineales de primer orden}
\label{chap:capitulo1}
\subfile{teoria/lineal}

\chapter{Sistemas de ecuaciones lineales de primer orden}
\label{chap:capitulo2}
\subfile{teoria/sistemas}


\chapter{Ecuación escalar de orden superior}
\label{chap:capitulo3}
\subfile{teoria/escalar}

\chapter{Análisis cualitativo de sistemas planos}
\label{chap:capitulo4}
\subfile{teoria/cualitativo}

\chapter{Series de potencias}
\label{chap:capitulo5}
\subfile{teoria/series}

\chapter{Transformada de Laplace}
\label{chap:capitulo6}
\subfile{teoria/laplace}

\begin{appendices}

\chapter{Problemas resueltos}

\section[Problemas del capítulo 1]{Problemas del capítulo~\ref{chap:capitulo1}}
\subfile{problemas/lineal}

\section[Problemas del capítulo 2]{Problemas del capítulo~\ref{chap:capitulo2}}
\subfile{problemas/sistemas}

\end{appendices}

\backmatter

\end{document}
