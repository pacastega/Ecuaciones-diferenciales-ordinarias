\documentclass[../main.tex]{subfiles}

\begin{document}

\begin{definition}
Las ecuaciones diferenciales lineales de primer orden son aquellas de la forma
\[x' = a(t)x + b(t)\]
donde \(a(t)\) y \(b(t)\) son continuas en \((\alpha, \omega) \subset \R\).
\end{definition}

\section{Ecuación diferencial lineal homogénea de primer orden}

Para resolver este tipo de ecuaciones primero consideraremos su ecuación
homogénea asociada, esto es
\[x' = a(t)x.\]

Buscamos ahora las soluciones de la ecuación homogénea. Supongamos que la
solución no se anula en ningún punto, entonces por el teorema de Bolzano
\(x > 0\) o \(x < 0\) en \((\alpha, \omega)\). En este caso recurrimos a un
método que llamaremos separación de variables, por el teorema fundamental del
cálculo podemos escribir
\[\frac{x'}{x} = a(t) \iff \log\abs{x(t)} = \int_{\beta}^t a(s) \dif s + c,
	\quad \beta \in (\alpha, \omega).\]
Para aligerar la notación escribiremos
\[\int^t a(s) \dif s,\]
para referirnos a cualquier primitiva de \(a\), y así obtenemos
\[x(t) = k e^{\int^t a(s) \dif s}.\]

Vamos a demostrar ahora que la solución así obtenida es la solución general de
la ecuación homogénea. Primero probamos que \(k e^{\int^t a(s) \dif s}\) es
solución de la ecuación homogénea, para ello basta derivar en
\[x'(t) = k a(t) e^{\int^t a(s) \dif s} = a(t) x(t).\]

Probamos ahora que no existen más soluciones, para ello consideramos
\[\Phi(t) = u(t) e^{-\int^t a(s) \dif s}\]
entonces
\begin{align*}
	\Phi'(t) &= u'(t) e^{-\int^t a(s) \dif s}
		- u(t) a(t) e^{-\int^t a(s) \dif s} \\
		&= a(t) u(t) e^{-\int^t a(s) \dif s} - u(t) a(t) e^{-\int^t a(s) \dif s}
		= 0
\end{align*}

Luego es constante, por lo que \(u(t) = k e^{\int^t a(s) \dif s}\).

\begin{example}
	Obtener la solución general de
	\[y' + 2ty = 0\]
\end{example}

\begin{solution}
	Tenemos entonces que
	\[\frac{y'}{y} = -2t \iff \log\abs{y} = -t^2 + c \iff y = k e^{-t^2}\]
\end{solution}

\subsection{Problema del valor inicial}

Tenemos una ecuación lineal homogénea de primer orden, esto es
\[x' = a(t)x\]
y además queremos que \(x(t_0) = x_0\) para algún \(t_0 \in (\alpha, \omega)\) y
\(x_0 \in \R\).

Sabemos que la solución general de la ecuación es de la forma
\[x(t) = k e^{\int^t a(s) \dif s},\]
y podemos usar la libertad que tenemos para elegir la primitiva para ponerla como
\[x(t) = k e^{\int_{t_0}^t a(s) \dif s},\]
ahora queremos que \(x(t_0) = x_0\) para ello notamos que
\[x(t_0) = k e^{\int_{t_0}^{t_0} a(s) \dif s} = k = x_0,\]
hemos probado así que el PVI tiene solución única, dada por
\[x(t) = x_0 e^{\int_{t_0}^t a(s) \dif s}\]
y además la solución está definida en \((\alpha, \omega)\).

\subsection{Estructura del espacio de soluciones}

Observamos que el conjunto de soluciones de una ecuación lineal homogénea de
primer orden es un espacio vectorial de dimensión 1, o más precisamente un
subespacio vectorial de dimensión 1 de \(C^1(\alpha, \omega)\), es decir, el
espacio vectorial de funciones de clase \(C^1\) en el intervalo
\((\alpha, \omega) \subset \R\).

Vamos a ver cómo esto se deduce de la linealidad de la ecuación. Para ello
definimos el operador diferencial:

\begin{align*}
  L : C^1(\alpha, \omega) &\to C(\alpha, \omega) \\
  x &\mapsto \frac{\dif x}{\dif t} - a(t)x
\end{align*}

Recordamos que decimos que una función es lineal cuando cumple
\[f(ax + by) = af(x) + bf(y), \quad a, b \in \mathbb{K},\]
entonces la linealidad de \(L\) se deduce de la linealidad de todas las
operaciones dentro de su definición.

Notamos que la linealidad de la ecuación diferencial equivale a la linealidad de
\(L\) y, además, \(x\) es solución de \(x' = a(t)x\) si y solo si
\(x \in \text{ker}(L)\). Sabemos del álgebra lineal que el núcleo de un
homomorfismo vectorial es siempre un subespacio vectorial del espacio de dominio
y, por tanto, el conjunto de soluciones de la ecuación homogénea es un espacio
vectorial.

\end{document}
